

\section{CBM detectors and their tasks}
Construction of the Compressed Baryonic Matter (\gls{CBM}) experiment is currently underway at \gls{FAIR}. Figure~\ref{fig:exp} depicts the CAD drawing of the \gls{CBM} experiment. The beam enters the experimental cave from the left side and traverses the High Acceptance Di-Electron Spectrometer (\gls{HADES}) experiment. 

\begin{figure}[!h]
    \centering
    \includegraphics[width=0.95\columnwidth]{Chapter1/images/CBMnew.png}
    \caption{HADES experiment on the left side and the \gls{CBM} experiment on the right side.}
    \label{fig:exp}
\end{figure}

The main features of the \gls{CBM} experiment are described below:
\begin{itemize}
\item tracking acceptance: $\SI{2.5}{\degree} < \theta_{lab} < \SI{25}{\degree}$,
\item peak intensities reach 10 MHz for Au+Au systems,
\item fast and radiation hard detectors,
\item free-streaming \gls{DAQ},
\item 4D tracking,
\item online event reconstruction and selection,
\item data rates up to 1~TB/s.
\end{itemize}


In order to get a valuable insight into the physics observables proper detector systems need to be developed. Usually, the detector concept is based on identifying the stable charged particles that are often decay products of collisions. Charged particles can be bent in a magnetic field of known strength to investigate their momentum. A Time-of-flight (\gls{TOF}) can discriminate heavier particles from the lighter ones by measuring the time of flight between two scintillators. Electrons are considered special due to their nature. They may contain undisturbed information from the early, dense phase of the fireball evolution. On the other hand, it's necessary to separate them from the abundant pions. In order to perform the complete analysis, the following detector systems are foreseen for the \gls{CBM}:


\textbf{Beam monitor, or T0 (\gls{BMON})} and its conceptual design, was summarized during the 40th \gls{CBM} Collaboration Meeting~\cite{bmon}. Two separate stations in front of the target are made out of high-purity poly-crystalline CVD diamond material. This detector is foreseen to monitor beam quality (position, time structure) and determine the start time of the reaction.\bigbreak

\textbf{Micro Vertex Detector} (\gls{MVD}) consists of four planar stations with Monolithic Active Pixel Sensor (\gls{MAPS}) chips. A station's layout and distance from the target can be tailored to the needs of a specific run, for example, to optimize vertexing or tracking. The vertexing detector geometry aims at a precision of secondary vertices determination of about 50 -- 100~$\mu m$ along the beam axis. The main aims cover decay vertex identification of the very short-lived particles such as charmed mesons, which decay within a few hundred $\mu m$ behind the target, as well as background rejection in di-electron spectroscopy~\cite{MVD}.\bigbreak

 \textbf{Silicon Tracking System} \gls{STS} which is responsible for tracking of charged particles and measuring their momentum. The \gls{STS} is located inside a superconducting dipole magnet~\cite{Malakhov:109025}. The charged particles traversing the active area of \gls{STS} are bent due to the applied magnetic field. The curvature depends on the momentum of the particles which allows for determining the latter value.\bigbreak
 
\textbf{Muon Chamber System} \gls{MUCH} is the fourth subsystem of the \gls{CBM} experiment. It is dedicated to muon detection (for example rare particles decaying into muons like $J/\psi$) and track reconstruction. This concept is based on layered design of the hadron absorbers (5 layers), separated with tracking detector planes which are based on Gas Electron Multiplication (\gls{GEM}) and Resistive Plate Chambers (\gls{RPC}) detectors~\cite{MUCH}.\bigbreak

\textbf{Ring Imaging Cherenkov Detector} \gls{RICH} is responsible for electron identification via Cherenkov radiation~\cite{RICH}. It allows separating electrons from pions up to 8 GeV/c. The detector consists of \SI{1.7}{\metre} long $CO_{2}$ gas radiators with pion threshold for Cherenkov radiation of 4.65 GeV/c, two arrays of mirrors and photon detector planes. Due to the substantial material budget in front of the RICH detector, central Au + Au collisions at 25 AGeV beam energy exhibit on the order of 100 rings. However, models predict that a pion suppression on the level of 500 will be accomplished thanks to the high granularity (about 55 000 channels) and high number of photons per ring~\cite{RICH}.\bigbreak

\textbf{Transition Radiation Detector} \gls{TRD} suppresses pions and  supports track reconstruction with 9-10 detector layers grouped into 3 stations. It is placed from approximately \SI{4}{\metre} to \SI{9}{\metre} downstream of the target, and the total active area reaches \SI{600}{\square\metre}. The working principle of the detector is based on the phenomenon that the ultra-relativistic particles traversing through a medium with different dielectric constant produce transition radiation. It is composed of two parts, the readout chamber and the radiator. The photons are generated by the electrons passing through the radiator, while the heavier pions don't produce any radiation. For detection of the produced radiation, multi-wire proportional chambers will be used, composed of a mixture of $\mathrm{Xe/CO_{2}}$ gases~\cite{TRD}. \bigbreak

\textbf{Time-of-Flight Detector} \gls{TOF} is designed to identify hadrons (pions, kaons, and protons). As the name indicated, the detector measures the time-of-flight of the reaction products with Multi-Gap Resistive-Plate Chambers (\gls{MRPC}). The \glspl{MRPC} have an excellent time resolution of \SI{60}{\pico\second}.  It will be located between \SI{6}{\metre} and \SI{10}{\metre} (depending on the physics objectives) and will cover an area of about \SI{120}{\square\metre}~\cite{TOF}. \bigbreak

\textbf{Projectile Spectator Detector} \gls{PSD} determines the collision centrality and event plane. The detector is meant to measure the nucleons from a projectile nucleus in the heavy ions collisions. The proposed 44 module design of the PSD covers a large transverse area around the beam spot position, such that most of the projectile spectator fragments deposit their energy in the \gls{PSD}. The detector concept is a compensating hadron calorimeter consisting of lead-scintillator sandwich modules with scintillator layers~\cite{PSD}.\bigbreak


The CBM detector system can be used in two operation modes: the first one is optimized for electron identification (electron configuration) and the second is specialized for muon identification (muon configuration). In the first one, all the subsystems apart from MUCH will be involved. In the muon configuration, the \gls{RICH} detector is replaced by \gls{MUCH}.

For the high-rate CBM experiment, the data read-out and acquisition system plays a crucial role. The time-stamped signals will be read out without event correlation and transferred to a high-performance computing farm, the GSI GreenIT Cube. The online event reconstruction and selection are performed by high-speed algorithms. In the first step, the tracks of the charged particles were reconstructed from the space and time information of the various detector signals. Subsequently, the particles will be identified, taking into account secondary decay vertices and information of \gls{RICH} or \gls{MUCH}, \gls{TRD}, and \gls{TOF}. Finally, the particles will be grouped into events, which will be selected for storage if they contain important observables. In parallel, the event is characterized using information from the PSD.

Another important online system is called Experiment Control System (\gls{ECS}) and is composed of software structure which aim to provide automatization, monitoring, and control of all the detector subsystems. A detailed description of \gls{ECS} and the detector specific control system (\gls{DCS}) will be given in Chapter~\ref{chap:online_systems}.



\section{Thesis overview and its rationale}
The thesis is divided into 7 chapters. The second chapter introduces the role of silicon trackers in large scientific experiments and brings the design details of the \gls{STS} closer, including the physics performance and experimental challenges. Those elements are directly connected with the requirements for the Detector Control System (\gls{DCS}). The third chapter serves as an introduction to how to control and monitor a large experiment, with an extended focus on the detector-related slow control system and the developed control framework. The next three chapters are focused on the results and their consequences for the experiment:
\begin{itemize}
    \item chapter 4 covers the first implementations of the mentioned control framework. The first application is related to the slow control interface for the \gls{FEE} readout. The second and third examples are related to the irradiation studies of the powering modules for the \gls{LV} powering of the \gls{STS} electronics and thermal cycling of \glspl{FEB}. The performed studies and results of these activities are discussed in detail,
    \item chapter 5 describes the efforts to design and test a distributed sensing system for the \gls{STS} with a focus on humidity sensing. Three considered technologies feature capacitive sensors, fiber optic sensors, and remote sensing with the use of a sampling system. The chapter focuses on the design choices and characterization of the fiber Bragg grating-based sensors. 
    \item chapter 6 focuses on the small-scale prototype version of the \gls{STS} in the \gls{mCBM} experiment. The first sections describe in detail the hardware and software solutions implemented in the detector. Subsequently, the results from the full-blown \gls{DCS} are presented and discussed, including the radiation effects on the silicon sensors and general considerations about the power dissipation of different elements of the \gls{STS}'s powering scheme. Moreover, considerations about the \gls{DCS} are given. 
\end{itemize}
The last part of the thesis summarizes the results and sheds a light on the next steps toward the realization of the \gls{STS} and its controls. The most important findings and results from the performed studies are also discussed.
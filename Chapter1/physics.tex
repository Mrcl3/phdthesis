Throughout the history, mankind has always been fascinated with the physics laws governing life on Earth. The theory known as atomism was one of the first attempts to understand the nature of matter. The concept of atoms formulated by Leucippus of Abdera (5th century before the common era) and further pursued by Democritus initiated the search for the building blocks of nature~\cite{sep-atomism-ancient}. According to this theory, everything consists of "atoms" (in ancient Greek \foreignlanguage{greek}{ἄτομος} which means uncuttable) that are physically indivisible and the rest of space is just void. 

It's astounding to consider that at the turn of the twentieth century (roughly, 2500 years after atomism was formulated), the structure of the atom remained unknown. The electron had just been discovered, and its behavior and characteristics were still poorly understood. The knowledge about the particle physics was scarce. Particles like nuclei, protons, neutrons were poorly understood~\cite{intro_particle_physics}. 
Further discoveries of radioactivity (in the year 1896) and radioactive elements (in the year 1898) by H. Becquerel and M. Skłodowska-Curie marked the gateway to 20th-century physics. 

High energy physics aims to explore the smallest and largest scales of the universe, seeking out new discoveries from the tiniest particles to the largest objects in space.
The number of important discoveries that have been made in the twentieth century is truly remarkable. Eventually, in 1970 these findings led to the formulation of a mathematical model called the Standard Model~\cite{intro_particle_physics}.  It describes the strong, weak, and electromagnetic fundamental \footnote{The interactions that are considered not to be reducible to more basic ones}{interactions} between the particles. 
 
 
 This work aims to bring the reader on a journey from the theoretical objectives of physics, to the often complex process of development and construction of a particles' detector for a high-energy physics experiment.
 %Modern ways of conducting experiments often involve use of advanced engineering and software engineering.
 
\section{Standard Model and the strong interaction}

The standard model is one of the most successful theories, which explained numerous results from experiments worldwide to date. The most notably predictions of the Standard Model are the Higgs boson, W and Z bosons, the gluon, and the top and charm quark.

The standard model contains 17 particles (see Figure~\ref{fig_standard})  that are the smallest building blocks and are categorized into two groups: bosons and fermions. These groups are distinguished by their spin difference: fermions (characterized by half-integer spin) obey Fermi–Dirac statistics and bosons (integer spin) obey Bose–Einstein statistics. 

Fermions are divided into two classes: quarks, which interact with the strong nuclear force, and leptons, which do not interact with it. Up and down quarks are located at the heart of atoms, inside the protons and neutrons. The other four quarks are only observed in particle accelerator collisions.

The electron is the most recognized of the leptons. Other charged leptons, known as muons and taus, are only discovered in particle accelerators and cosmic rays from space. Furthermore, each of the mentioned leptons has its corresponding neutrino, which has no electrical charge and a very small mass.

In addition to the particles, the Standard Model includes three forces that govern the behavior of matter. These forces are electromagnetism, strong and weak nuclear forces. The force transmitting particles are the photon (electromagnetism), the gluon (strong nuclear force), the W \& Z bosons (the weak force), and the Higgs boson.

\begin{figure}[!h]
\centering
\includegraphics[width=0.7\columnwidth]{Chapter1/images/particles.png}
\caption{Fermions and bosons of the Standard Model~\cite{standard_model}.}
\label{fig_standard}
\end{figure}

\newpage

The Strong Force's theoretical foundation, Quantum Chromodynamics (QCD), is well established, with quarks and gluons describing the interaction between quarks mediated by gluons.

Yet, key phenomena in strong interactions, like the confinement of quarks and gluons into \footnote{Hadrons are composed of quarks, and therefore they experience the strong nuclear force}{hadrons} and the creation of mass, have remained unclear.

In extreme environments such as the high-temperature T, or the high baryon density $\rho$, the confined baryonic matter may form a new state of matter called the quark-gluon plasma (\gls{QGP})~\cite{phase_diagram}. Hot deconfined matter dominated the early cosmos just a few microseconds after the Big Bang, but compact stars may also contain cold and baryon-rich quark matter in their interiors.

A well-established non-perturbative approach to solving the quantum chromodynamics theory of quarks and gluons is known as the Lattice \gls{QCD}. LQCD can also be used to address issues like the mechanism for confinement and chiral symmetry breaking, the role of topology, and the equilibrium properties of \gls{QCD} at finite temperature~\cite{lattice_qcd}. 

\section{Studies of nuclear matter and its forms}
The water phase diagram is a pressure-temperature representation of various water phases such as water, steam, and ice. It also describes the borders between these states and types of transitions. 

Similarly, it's possible to create such a diagram for the nuclear matter, what is depicted in Figure~\ref{fig_phase}. It can be summarized as a combination of the experimental results and theoretical predictions for the behavior of the nuclear matter.

The experimental search for a quark-gluon plasma in heavy ion
collisions was shaped by several model predictions of possible \gls{QGP} signals: suppressed production of charmonium states (bound states of a charmed quark and a charmed antiquark), in particular $\mathrm{J/\psi}$ mesons, enhanced production of strange and multi-strange hadrons from the \gls{QGP}, characteristic radiation of photons and dilepton pairs from the \gls{QGP}. The results of the QGP search programme at the \gls{CERN} \gls{SPS} on center collisions of medium and heavy nuclei found to be adequate with the QGP predictions.

The crossover transition region was experimentally investigated at the Large Hadron Collider (\gls{LHC}) and Relativistic Heavy Ioc Colllider (\gls{RHIC}). This region is characterized by a low baryon chemical potential and high temperatures (around 150~MeV), at which the ratio of quarks to antiquarks is almost equal. Calculations of lattice \gls{QCD} show that the transition at $\mu_{B} = 0$ is a smooth crossover~\cite{Aoki_2006} for
physical quark masses. Because there is no real phase transition, the crossover temperature is unclear, as alternative definitions might result in different results. The matter is expected to hadronize at temperature of 155--160 MeV~\cite{Bazavov_2012, Stachel_2014}.



\begin{figure}[!h]
\centering
 \includegraphics[width=0.65\columnwidth]{Chapter1/images/phase.png}
\caption{The phase diagram illustrating the regimes of confined and deconfined nuclear matter. The critical point separates the region of a cross-over (explored by RHIC and LHC) from that of a first-order phase transition to be studied by the CBM experiment~\cite{friese_diagram}.}
\label{fig_phase}
\end{figure}
\newpage
Figure \ref{fig_phase} also depicts structures at higher baryon-chemical potentials, such as a chiral and a deconfinement first-order phase transition merging at a critical point with a region of quarkyonic matter in between. To date, none of these structures or phases have been discovered. As previously stated, first-principles theories, such as perturbative QCD, continue to struggle to generate solid predictions for matter characteristics at larger baryon-chemical potentials~\cite{Sakai_2008, Fischer_01, Tawfik_01}. 


%\section{Physics goa}

\section{Probing dense nuclear matter with heavy ion collisions}

Heating or compressing nuclear matter leads to deconfined matter when the temperature and density exceed a threshold point, as denoted in Figure~\ref{fig_phase}. Heavy ion collisions in the energies between 2~AGeV and 11~AGev have an enormous potential to explore many aspects of the phase diagram. Figure \ref{fig:cbm_density} represents the time evolution of a Au+Au collision system at 10~AGeV. Such conditions are expected to take place during the supernovae core collapse and in the core of neutrons stars. Furthermore, the calculations of different transport models and hydrodynamics show that the density of the fireball will reach more than $8\rho_{0}$ during the Au+Au collisions at 10~AGeV~\cite{CBM_physics}.
\newpage
\begin{figure}[!h]
    \centering
    \includegraphics[width=0.65\columnwidth]{Chapter1/images/CBM_density.png}
    \caption{The time evolution of the central net baryon density $\rho(t)$ (top) calculated using different transport models and 3-fluid hydrodynamics of a head-on Au+Au collision at 10 AGeV energy (indicated in GeV/A)~\cite{CBM_physics}.}
    \label{fig:cbm_density}
\end{figure}

Figure~\ref{fig_heavyion} depicts the presumed evolution of the heavy ion collision. As mentioned before, the creation of the \gls{QGP} depends mostly on the conditions (temperature and pressure) of the colliding particles. It illustrates the various forms of QCD matter intervening during the subsequent phases of the collision for hadronic collisions and heavy ion collisions.
\begin{figure}[!h]
\centering
 \includegraphics[width=0.6\columnwidth]{Chapter1/images/heavyion.png}
\caption{Schematic representation of the various stages of a HIC as a function of time t and the longitudinal coordinate z (the collision axis)~\cite{Sahoo:2745520}.}
\label{fig_heavyion}
\end{figure}

Figure \ref{fig_heavyion} depicts two possibilities, with and without \gls{QGP}. The critical temperature is represented by $T_c$, whilst the freeze-out and chemical freeze-out temperatures are indicated by $T_{fo}$ and $T_{ch}$, respectively. Following the collision (right side of the graph in Figure~\ref{fig_heavyion}), the system enters a pre-equilibrium phase, followed by a deconfined QGP medium and a probable mixed phase (which should exhibit first order phase transition signals). Evolution of such a system usually takes between 10 and 100~fm/c. 

The heavy ion experiments around the world have been exploring the facets of the phase diagram in many energies and/or density ranges. Figure~\ref{fig:cbm_rates} shows the interaction rates of existing and planned heavy-ion experiments. Groundbreaking heavy-ion experiments at AGS at Brookhaven and at low CERN-SPS beam energies have investigated the QCD phase diagram at large baryon chemical potentials. Because of the detector technologies available at the time, these observations were limited to abundantly generated hadrons and di-electron spectra with severely constrained statistics. The NA61/SHINE experiment at CERN-SPS has been searching for the first-order phase transition by monitoring hadrons with light and heavy ion beams~\cite{CBM_physics, Turko:2301677}.

The studies conducted at the Solenoidal Tracker at \gls{RHIC} (\gls{STAR}) and A Large Ion Collider Experiment (\gls{ALICE}) revealed that the partonic degrees of freedom prevail at the early phase of the \footnote{Notation for a large volume \SI{1}{\femto\cubic\metre} system}{fireball} evolution~\cite{CBM_physics}.

The HADES detector at SIS18 detects hadrons and electron pairs in heavy-ion collision systems at reaction rates of up to 20kHz and beam energy of 1-2 AGeV~\cite{Ablyazimov_2017}.


The STAR Collaboration at \gls{RHIC} has
performed a beam energy scan from top energies down
to $\sqrt{s_{NN}} = 7.7$~GeV. The BES phase I program findings indicate evidence of a first-order phase transition in the QCD phase diagram and the turn-off of the quark gluon plasma's distinctive fingerprints at low collision energy. The TPC read-out limits the reaction rates of STAR to around 800Hz for beam energies over $\sqrt{s_{NN}} = 20$~GeV, and drops to a few Hz at beam energies below $\sqrt{s_{NN}} = 8$~GeV due to the diminishing beam luminosity delivered by the \gls{RHIC} accelerator. BES phase II program (BES II) covers the $\sqrt{s_{NN}}$ from 7.7 to 19.6~GeV in the collider mode and from 3 to 7.7~GeV in the fixed-target mode~\cite{STAR2, STAR1}.

The Nuclotron at the Joint Institute for Nuclear Research (JINR) in Dubna is preparing the fixed-target experiment BM@N to explore heavy-ion collisions at gold beam energy up to roughly 4~AGeV. Furthermore, the Nuclotron-based Ion Collider fAcility NICA with the Multi-Purpose Detector (MPD) is being built at JINR. The NICA collider is intended to operate at collision energies ranging from $\sqrt{s_{NN}} = 8$ to 11 GeV, corresponding to a reaction rate of 6~kHz for minimal bias Au+Au collisions~\cite{Ablyazimov_2017}.

\begin{figure}[!h]
    \centering
    \includegraphics[width=0.7\columnwidth]{Chapter1/images/interaction_rates.png}
    \caption{Interaction rates achieved by existing and planned heavy-ion experiments as a function of the center-of-mass energy. “STAR FXT” denotes the fixed-target operation of STAR.  Blue symbols show collider experiments, black and grey
symbols show fixed-target experiments~\cite{Ablyazimov_2017}.}
    \label{fig:cbm_rates}
\end{figure}




\gls{CBM} is a fixed target experiment that aims to measure rare particles as probes of dense matter with very good precision at beam energies up to 11~AGeV or $\sqrt{s_{NN}}$ = 4.9 GeV (up to 14 AGeV for light nuclei and 29 AGeV for protons) and interaction intensities up to 10~MHz.
\newpage
It is well positioned to explore many facets of \gls{QCD} matter and to discover new exotic states. In addition, it will be able to explore the equation of state at densities and temperatures close to those probed by neutron star mergers. Majority of the experimental observables which are sensitive to the properties of dense nuclear matter, like the flow of identified (anti-) particles, higher moments of event-by-event multiplicity distributions of conserved quantities, multi-strange (anti-) hyperons, di-leptons, and particles containing charm quarks are prone to the statistics. The key feature of successful experiments is high rate capability, that ensure high precision~\cite{Ablyazimov_2017}. 

The \gls{CBM} aims to investigate the following:
\begin{enumerate}
    \item the equation of state of baryonic matter at neutron star densities,
    \begin{itemize}
        \item collective behavior and flow anisotropies - the collective hadrons motion provides the information on the dense stage of the heavy ion collision. It's driven by the pressure gradient created the at the beginning of the fireball evolution~\cite{Reisdorf_2007},
        \item hyperons and their interactions - are preferentially produced in the dense phase of the fireball via sequential collisions.
    \end{itemize}
    \item modifications of hadron properties in dense baryonic matter and the onset of chiral symmetry restoration. These phenomenons affect the invariant-mass spectra of di-leptons, which will be measured both in the electron and the muon channel,
    \item phase transitions from hydronic matter to quarkonic or partonic matter:
    \begin{itemize}
        \item the excitation function of multi-strange hyperons, which are driven into equilibrium at the phase boundary,
        \item the excitation function of the invariant mass spectra of lepton pairs which reflect the fireball temperature, and, hence, may reveal a caloric curve and a first-order phase transition,
        \item the excitation function of higher-order event-by-event fluctuations of conserved quantities such as strangeness, charge, and baryon number are expected to occur in the vicinity of the critical point.
    \end{itemize}

     
     
    \item hypernuclei (double $\lambda$, strange di-baryons etc.) and the measurement of their life time will provide information on the hyperon-nucleon and hyperon-hyperon interaction,
    \item charm production mechanisms.
\end{enumerate}
A detailed description and explanation of the \gls{CBM} physics program can be found in the \gls{CBM} Physics Book~\cite{CBM_physics} and in the summary paper~
\cite{Ablyazimov_2017}.




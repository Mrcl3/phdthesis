
 The discovery of radioactive elements and radioactivity was made by the M. Skłodowska-Curie and H. Becquerel. Those discoveries paved the way to an unimaginable amount of technological advances. As exotic and distant as they initially seem, fundamental studies expand our knowledge of the universe and teach its underlying principles.
 High energy physics explores the smallest and largest scales of the universe, seeking out new discoveries from the tiniest particles to the largest objects in the space. This quest inspires young minds, trains an expert workforce, and drives innovation that improves the nation’s health, wealth, and security.
 This work aims to bring the reader for a journey from the theoretical objectives of the experimental science, to the complex and sophisticated world of research and developments. 
 %Modern ways of conducting experiments often involve use of advanced engineering and software engineering.
 
\section{Standard Model}
The fundamental particles in the universe are classified in the Standard Model as fermions (matter particles) and bosons (force-carrying particles). There are three generations of fermions, but ordinary matter is made only from the first fermion generation. The first generation consists of up and down quarks which form protons and neutrons, and electrons and electron neutrinos. The three fundamental interactions known to be mediated by bosons are electromagnetism, the weak interaction, and the strong interaction
\section{Studying cosmic matter in the laboratory}

%\section{Physics goa}


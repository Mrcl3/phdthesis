\section{Controlling a detector}

Data acquisition, parameter monitoring, control, logging, and storage are compulsory tasks for every experimental setup, especially in remote locations with elevated radiation levels. To ensure the safe operation of a detector subsystem automation processes are commonly implemented (e.g. in form of a Finite State Machine \gls{FSM} or hardware interlocks). In the case of \gls{STS}, to ease the use and implementation of a control system, a fairly novel approach was used. It is primarily based on the \footnote{Containerization is the packaging of software code with just the operating system libraries and dependencies required to run the code.}{containerization} of different applications used to monitor and control setups. 

%\begin{figure}[!h]
%\centering
%\includegraphics[width=0.55\columnwidth]{Chapter3/Controls/images/example.png}
%\caption{General detector control system architecture}
%\label{fig_DCS_arch}
%\end{figure}
The control system must provide not only means of communicating between the hardware and software layers but also visualization, logging, archiving, and controlling means (either in an automated way by a Finite State Machine (\gls{FSM}) or manually). The control system can be usually divided into three layers: field layer, control layer, and supervisory layer. The bottom (field) layer contains all the process sensors, actuators, and other devices that are connected to the control system via I/O boards and/or field buses. Communication between the field layer and control layer can be of almost any type compatible with the used components, i.e. Ethernet, Modbus TCP, Profibus. The control logic is introduced in \glspl{PLC} and so-called control nodes (single board computers etc.) in the control layer. The supervision layer or supervisory level provides the operators with means of controlling and monitoring the subsystem, for example via command line or a Graphical User Interface (\gls{GUI}) or Operator Interface (\gls{OPI}) \cite{layers}.  Typically, DCS's building blocks reside in a dedicated network to avoid unnecessary cross-talk and ensure more efficient debugging.

Figure \ref{fig_arch} shows a general idea behind the \gls{STS}'s \gls{DCS} from the software point of view.  The master node or the central \gls{DCS} node gets the data from the configuration database, which will allow the preparation of subsystems for a given action (for example for a transition into a different state). The master node will be only accessible by the \gls{DCS} experts, excluding subsystem-related personnel from performing actions on other subsystems' \gls{DCS}. There are also one or more archiving nodes and control nodes, which will contain detector-specific applications. All the mentioned components allow effective control over a detector and deliver crucial operational information (alarms, events, \glspl{PV} values, etc.). 

\begin{figure}[!h]
\centering
\includegraphics[width=1\columnwidth]{Chapter3/Controls/images/DCS.png}
\caption{Proposed \gls{DCS} infrastructure for the \gls{STS}}
\label{fig_arch}
\end{figure}
\newpage






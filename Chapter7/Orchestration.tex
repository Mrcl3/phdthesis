\subsection{Outlook for the DCS software services} 
Weavescope used for the \gls{mSTS} provides basic monitoring and handling functionalities for the containers based systems. The metrics proved to be extremely useful in detecting issues and managing containers. Nevertheless, for the final experiment, a more sophisticated software infrastructure is foreseen. It's necessary to address it by automating deployment. Similarly, as for the containers, an orchestrator needs to be secure, scalable, highly available, provide logging, and monitoring capabilities, and be redundant. Orchestration leads to automation of the container deployment, as well as to balance the workload. Orchestrators can:
 \begin{itemize}
     \item automatically deploy containers based on policies, application load, and environmental metrics,
     \item identify failed containers or clusters and heal them,
     \item manage application configuration,
     \item connect containers to storage and manage networking,
     \item improve security by restricting access in between containers, and between containers and external systems.
 \end{itemize}

 Examples of orchestrators include DockerSwarm~\cite{DockerSwarm}, and Kubernetes \cite{Kubernetes}. One of the recent applications of the Kubernetes was reported in \cite{ICALEPCS2021:Diamond}, where the whole test beamline is operated with containers. Similarly, the whole \gls{CBM}'s \gls{DCS} should be operated with containers and orchestration tools. 

 Apart from the orchestration, the \gls{DCS} should be supplemented with:
 \begin{itemize}
      \item gateway - let's consider the low voltage powering of the STS's \gls{FEE}, which consists of about 2100 low voltage channels, 140 modules, and 14 crates. Each crate has a controller with an embedded \gls{IOC}, publishing all the process variables. By putting the power supplies into a different subnet, we can not only easily debug potential issues, but also limit the network traffic. That's why it has been endorsed to use CA Gateway \cite{gateway} or \gls{PV} gateway (to be tested), which will take care of regulating access between the subnets in the DCS network. It also provides additional access security, assuring that the \glspl{IOC} running the key services like powering run smoothly.
     \item time synchronization - as mentioned in the section \ref{archiver}, the archiver could be split into a few nodes serving as temporary data storage (short-term, mid-term). Proper daily backup to GSI managed database would be recommended, but it depends mostly on the database services provided by the GSI IT. So far, the Redis DB has been used, but also other options should be considered.       
     \item logbook - One of the missing elements in the \gls{mSTS}'s architecture is a logbook. So far, for all the mCBM-related activities, elog \cite{elog} was used. For the final experiment, a dedicated elog branch will be implemented and the elog client will be used \cite{elog_client},
     \item save and restore service - it is organized twofold. A tool, called autosave, which is a part of the synApps module \cite{autosave} preserves \glspl{PV} values through the \gls{IOC} reboot. The second set of tools that permits taking snapshots and saving configurations is MASAR \cite{masar}. It's a more complex tool than autosave, offering also a Phoebus-based \gls{GUI},
     \item data persistence - to properly archive and analyze the data, it's necessary that all nodes, \glspl{IOC}, and other software applications are synchronized. As it's impossible to adjust the clock in the containers, it should be synchronized on each node separately. The central \gls{DCS} node will provide a Network Time Protocol (\gls{NTP}) daemon that will be synchronized with one of the public, official sources.  By doing so, the clocks of all the containers running control applications will be automatically synchronized even if the external network connection is not available,
     \item communication protocol - it was reported during the EPICS Collaboration meeting 2022 \cite{epics_2022} that the transition to the newer protocol (PV access) is ongoing. Nevertheless, CA and PVA are both included in the EPICS 7, which should be the base image of the \gls{CBM} \gls{IOC} image, and also for the next versions of the \gls{IOC}. PVA is under constant development and will offer even more features in the coming years, thus for the future CBM experiment (timeline of more than 10 years), it's an optimal choice. 
 \end{itemize}


\subsection{Failover considerations}
The high availability of services plays a key role in the safe operation of a detector. Once all the services are deployed, only minimal operation breaks are foreseen during 10 years of operation. The STS needs to be constantly cooled, in order to avoid performance degradation of the silicon sensors.
Crucial elements of the STS, like the air drying plant or the cooling plant, will be monitored and controlled by a PLC-based system. This hardware layer will provide essential safety measures in case of failure. The PLC will also be linked to an \gls{IOC} publishing the values to the software layer. Even before triggering the hardware interlock, any potentially hazardous system behavior will be discovered by the software of the control system. In order to ensure maximum safety, failover mechanisms will be exercised to mitiage potential \gls{IOC} failures. There are 3 considered methods to address failover. 

Suppose the hardware is controlled, e.g. via a network. In that case, it's possible to deploy a backup \gls{IOC}, which has the same configuration as the main \gls{IOC}, therefore providing a replacement if it fails. Under some standards, like example RS232, it's not considered a good practice to have two \glspl{IOC} connected to the same node. Furthermore, in the case of RS232 a multiplexer would be needed to implement a redundant solution. 

A second possibility is to use failover mechanisms based on an orchestrator, in this case, the deployment and life cycle of a container is governed by an additional tool, i.e. Kubernetes. In case one of the containers (\glspl{IOC}) hangs up, it will be automatically stopped, and a new container will take over the tasks. Nevertheless, the newly deployed container could have a different configuration, therefore changing the state of the whole system. 

\section{Failure considerations}

Failure considerations are an essential aspect of designing a detector and its services. Some systems might need to be planned as redundant structures that bring a detector into a safe state in case of an issue and/or failure. There are a few layers of dependencies and interlocks, including the software and hardware ones.

Low voltage power supply failures can be divided into channel, module, controller, or crate failures. The example below describes the worst-case scenario in which the failed low-voltage module was connected only to the readout boards.
\begin{itemize}
    \item channel failure - in case of a permanent failure, up to 5 \glspl{FEB} might be off (if that channel powers a \gls{ROB}),
    \item module failure - up to 75 \glspl{FEB} might get disabled (4.3\% of all boards), 
    \item crate failure (10 \gls{LV} modules in the crate) - up to 750 boards stop working, which constitutes to about 43\% of the STS.
\end{itemize}

As the LV power supplies will be located in the array which is not accessible during the beam time, any kind of permanent failure will result in a loss of the physics data. 

Another type of failure that is dangerous for the system is an uncontrolled increase of water content inside the detector. This could be caused by the failure of the drying system, loss of confinement, or cooling system failure. 
\section{Detector safety}


\section{Outlook and summary}

The scope of this work focused on developing a modular control system framework which can be implemented for small, medium and large experimental setups. This framework was used for setups that required a remote operation, like the irradiation of the powering modules for the \gls{FEE}, but also in laboratory-based setups where the automation and archiving was needed (thermal cycling of the \gls{STS} electronics).
With the help of the \gls{EPICS} related applications, it was found that the low voltage powering module will experience soft errors of up to 9 per month during the \gls{CBM} operation. Such behavior poses a risk to the experiment operation as it could cause deterioration of the physics performance, but also possible danger to the \gls{FEE}. On the other hand, the \gls{HV} channels would be switched off even more often, but in the case of the \gls{CBM} they are located far away from the experimental site.

It was further assessed what are the limitations of the \glspl{FEB} with respect to the thermal cycling. The results served as an indication of possible failure modes of the \gls{FEB} in the end of lifetime  experimental conditions. Limitations in the performance of the boards were found, and potential reasons determined (e.g. \gls{CTE} difference between the materials). 

Another application of the developed framework was related to the testing and characterization of the humidity sensors. A general strategy for ambient parameters monitoring inside the \gls{STS} was developed, and potential candidates were chosen. Sniffing system with a ceramic sensor and \gls{FOS} were identified as reliable solutions for the \gls{STS}. 
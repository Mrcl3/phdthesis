The first section of the last chapter covers the next major milestone to be completed before the assembly of the final system start. The next section gives an outlook for the \gls{STS}'s and the foreseen developments of the \gls{DCS}. At last, the major takeaways of the thesis will be highlighted and shortly described.

\section{Thermal Demonstrator}
Another important milestone toward the assembly of the \gls{STS} is the thermal demonstrator. 
It is tasked to implement the silicon sensor and electronics concept under realistic mechanical boundary conditions and to experimentally demonstrate the feasibility of \gls{STS}’s cooling concepts. Figure~\ref{fig:demo} depicts the thermal demonstrator with the respective dummy detector modules. These objects dissipate similar values of heat as the actual detector module. The excess heat needs to be evacuated in order to prevent the silicon sensors from thermal runaway and reverse annealing. 
\begin{figure}[!h]
    \centering
    \includegraphics[width=0.65\columnwidth]{Chapter7/images/thermal_demo.png}
    \caption{Schematics of the thermal demonstrator~\cite{thermal_demo}. The dummy modules are mounted onto carbon ladders and placed in the C-frames.}
    \label{fig:demo}
\end{figure}
\label{demo}
 The thermal demonstrator serves not only to demonstrate and evaluate the cooling of the \gls{STS}, but also to perform long-term humidity and dew point measurements with the \gls{FOS} and sniffing system described in the \ref{chap:fos}. As the temperature inside this demonstrator will reach the end of lifetime temperature in the \gls{STS}, it's a unique opportunity to test the mentioned sensors. 
Due to the low temperatures inside the system, the frost point needs to be kept at values below \SI{-45}{\celsius}. To achieve this, a dedicated air drying system was developed and will be tested together with all the other services. 

 Apart from the humidity sensors, the thermal demonstrator features also temperature sensors:
 \begin{itemize}
     \item in order to evaluate the performance of the cooling plate and the thermal interfaces, each dummy \gls{FEB} is populated with 3 temperature sensors,
     \item to assess how much heat should be dissipated by the dummy silicon sensors, there are also temperature sensors placed on each dummy.
 \end{itemize}

A few services including the cooling plant that delivers the liquid to the test objects, humidity sensors but also interlocks and control strategies can be developed and then adjusted for the \gls{STS}. Hence, the thermal demonstrator is also a perfect opportunity to develop control system applications that will reduce the commissioning time of the \gls{STS}. In this case, further improvements of the containerized framework should be considered, e.g. \footnote{Orchestration is the coordination and management of computer systems, applications and/or services.}{orchestration} of the containers. The next sections focus on the further improvements of the \gls{DCS} in order to have a reliable and easy to maintain solution for the future. 
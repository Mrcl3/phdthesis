The Compressed Baryonic Matter (CBM) is one of the core experiments at the future Facility for Anti-proton and Ion Research (FAIR), Darmstadt, Germany. Its goal is to investigate nuclear matter characteristics at high net-baryon concentrations and moderate temperatures. The Silicon Tracking System (STS) is a central detector system of CBM. It is placed inside a 1\,Tm magnet and operated at temperature of about \SI{-10}{\celsius} to keep radiation-induced bulk current in the \SI{300}{\micro\metre} double-sided microstrip silicon sensors low. The design of the STS aims to minimize the material budget in the detector acceptance ($\SI{2.5}{\degree} < \theta < \SI{25}{\degree}$). In order to do so, the readout electronics is placed outside the active area, and the analog signals are transported via ultra-thin micro-cables. The STS comprises eight tracking stations with 876 modules. Each module is assembled on a carbon fiber ladder, which is subsequently mounted in the C-shaped aluminum frame. 

The scope of the thesis focused on developing a modular control system framework that can be implemented for different sizes of experimental setups. The developed framework was used for setups that required a remote operation, like the irradiation of the powering modules for the \gls{FEE}, but also in laboratory-based setups where the automation and archiving were needed (thermal cycling of the \gls{STS} electronics).

The low voltage powering modules will be placed in the vicinity of the experiment, therefore they will experience a total dose of up to 40\,mGy over the 10 years of \gls{STS} lifetime. To estimate the effects of the radiation on the low-voltage module performance, a dedicated irradiation campaign took place. It aimed at estimating the rate of radiation-induced soft errors, that lead to the switch off of the front-end electronics (\gls{FEE}).

Regular power cycles of multiple frond-end boards (\glspl{FEB}) pose a risk to the experiment operation.  Firstly, such behavior could negatively influence the physics performance, but also have deteriorating effects on the hardware. It was further assessed what are the limitations of the \glspl{FEB} with respect to the thermal cycling and the mechanical stress. The results served as an indication of possible failure modes of the \gls{FEB} at the end of \gls{STS} lifetime. Failure modes after repeated cycles, and potential reasons were determined (e.g., \gls{CTE} difference between the materials). 

Due to the conditions inside the \gls{STS} efficient temperature and humidity monitoring and control are required to avoid icing or water condensation on the electronics or silicon sensors. The most important properties of a suitable sensor candidate are resilience to the magnetic field, ionizing radiation tolerance, and fairly small size.

A general strategy for ambient parameters monitoring inside the \gls{STS} was developed, and potential sensor candidates were chosen. To characterize the chosen relative humidity sensors the developed control framework was introduced. A sampling system with a ceramic sensor and Fiber Optic Sensors (\gls{FOS}) were identified as reliable solutions for the distributed sensing system. Additionally, the industrial capacitive sensors will be used as a reference during the commissioning.

Two different designs of \gls{FOS} were tested: a hygrometer and 5 sensors multiplexed in an array. The \gls{FOS} hygrometer turned out to be a more reliable solution. One of the possible reasons for a worse performance is a relatively low distance between the subsequent sensors (15\,cm) and a thicker coating. The results obtained from the time response study pointed out that the thinner coating of about \SI{15}{\micro\metre} should be a good compromise between the humidity sensitivity and the time response. 

The implementation of the containerized-based control system framework for the \gls{mSTS} is described in details. The deployed EPICS based framework proved to be a reliable solution and ensured the safety of the detector for almost 1.5 years. Moreover, the data related to the performance of the detector modules were analyzed and significant progress in the quality of modules was noted. Obtained data was also used to estimate the total fluence, which was based on the leakage current changes. 

The developed framework provided a unique opportunity to automate and control different experimental setups which provided crucial data for the \gls{STS}. Furthermore, the work underlines the importance of such a system and outlines the next steps toward the realization of a reliable Detector Control System for \gls{STS}.
The Compressed Baryonic Matter (CBM) is one of the core experiments at the future Facility for Anti-proton and Ion Research (FAIR), Darmstadt, Germany. The Silicon Tracking System (STS) is a central detector system of CBM, placed inside a 1~Tm magnet and with an operation temperature of about \SI{-10}{\celsius} to keep low radiation-induced bulk current in the \SI{300}{\micro\metre} double sided microstrip silicon sensors.

The STS comprises eight tracking stations with 876 modules. Each module is calibrated and tested in order to access its performance. Next steps involve mounting the module on a carbon ladder, and subsequently these objects are arranged horizontally on so-called C-frames.

The scope of the thesis focused on developing a modular control system framework that can be implemented for small, medium, and large experimental setups. This framework was used for setups that required a remote operation, like the irradiation of the powering modules for the \gls{FEE}, but also in laboratory-based setups where the automation and archiving were needed (thermal cycling of the \gls{STS} electronics).

With the help of the \gls{EPICS} related applications, it was found that the low voltage powering module will experience soft errors of up to 9 per month during the \gls{CBM} operation. Such behavior poses a risk to the experiment operation as it could cause deterioration of the physics performance, but also a possible danger to the \gls{FEE}. On the other hand, the \gls{HV} channels would be switched off even more often, but in the case of the \gls{CBM} they are located far away from the experimental site.

It was further assessed what are the limitations of the \glspl{FEB} with respect to the thermal cycling and the mechanical stress that is therefore induced. The results served as an indication of possible failure modes of the \gls{FEB} at the end of \gls{STS} lifetime. Failure modes after repeated cycles, and potential reasons were determined (e.g., \gls{CTE} difference between the materials). 

Another application of the developed framework was related to the testing and characterization of the humidity sensors. A general strategy for ambient parameters monitoring inside the \gls{STS} was developed, and potential sensor candidates were chosen. A sampling system with a ceramic sensor and \gls{FOS} were identified as reliable solutions for the distributed sensing system. Additionally, the industrial capacitive sensors will be used as a reference during the commissioning.

The \gls{FOS} hygrometer turned out to be a more reliable solution in comparison to a sensor array. One of the possible reasons of  worse performance is a relatively low distance between the subsequent sensors and a thicker coating. The results obtained from the time response study pointed out that the thinner coating of about 15\,$\mu m$ should be a good compromise between the humidity sensitivity and the time response. 

Chapter~\ref{chap:msts} focused on the main implementation of the containerized-based control system framework for the \gls{mSTS}. The deployed system proved to be a reliable solution and ensured the safety of the detector for almost 1.5 years. Moreover, the data related to the performance of the detector modules were analyzed and significant progress in the quality of modules is observed. Obtained data was also used to estimate the total fluence, which was based on the leakage current changes. 
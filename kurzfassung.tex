Das Compressed Baryonic Matter (\gls{CBM}) Experiment ist eines der Kernexperimente an der zukünftigen Facility for Anti-proton and Ion Research (FAIR) in Darmstadt, Deutschland. Sein Ziel ist es, die Eigenschaften von Kernmaterie bei hohen Nettobaryonenkonzentrationen und moderaten Temperaturen zu untersuchen. Das Silicon Tracking System (\gls{STS}) ist ein zentrales Detektorsystem des \gls{CBM}, das in einem 1\~Tm-Magneten untergebracht ist und bei einer Betriebstemperatur von etwa \SI{-10}{\celsius} arbeitet, um den strahlungsinduzierten Volumenstrom in den \SI{300}{\micro\metre} doppelseitigen Silizium-Mikrostreifensensoren niedrig zu halten. Das Design des \gls{STS} zielt darauf ab, das Materialbudget in der Detektorakzeptanz ($\SI{2.5}{\degree} < \theta < \SI{25}{\degree}$) zu minimieren. Dazu wird die Ausleseelektronik außerhalb des aktiven Bereichs platziert, und die analogen Signale werden über ultradünne Mikrokabel transportiert. Das \gls{STS} besteht aus acht Tracking-Stationen mit 876 Modulen. Jedes Modul ist auf einer Kohlefaserleiter montiert, die anschließend in den C-förmigen Aluminiumrahmen eingebaut wird. 

Der Schwerpunkt der Arbeit lag auf der Entwicklung eines modularen Steuerungssystems, das für verschiedene Größen von Versuchsaufbauten eingesetzt werden kann. Der entwickelte Rahmen wurde für Versuchsaufbauten verwendet, die eine Fernsteuerung erfordern, wie die Bestrahlung der Stromversorgungsmodule für die \gls{FEE}, aber auch in laborgestützten Versuchsaufbauten, bei denen die Automatisierung und Archivierung erforderlich ist (thermische Zyklen der \gls{STS}-Elektronik).

Die Niederspannungsstromversorgungsmodule werden in der Nähe des Experiments platziert, so dass sie während der 10-jährigen Lebensdauer des \gls{STS} einer Gesamtdosis von bis zu 40\,mGy ausgesetzt sein werden. Um die Auswirkungen der Strahlung auf die Leistung der Niederspannungsmodule abzuschätzen, wurde eine spezielle Bestrahlungskampagne durchgeführt. Sie zielte darauf ab, die Rate der strahleninduzierten weichen Fehler abzuschätzen, die zum Abschalten der Frontend-Elektronik (\gls{FEE}) führen.

Regelmäßige Stromversorgungszyklen mehrerer Front-End-Platinen (\glspl{FEB}) stellen ein Risiko für den Betrieb des Experiments dar.  Zum einen könnte sich ein solches Verhalten negativ auf die physikalische Leistung auswirken, zum anderen aber auch zu einer Verschlechterung der Hardware führen. Ferner wurde untersucht, wo die Grenzen des \glspl{FEB} in Bezug auf thermische Zyklen und mechanische Belastung liegen. Die Ergebnisse dienten als Hinweis auf mögliche Versagensarten des \gls{FEB} am Ende der Lebensdauer des \gls{STS}. Die Versagensarten nach wiederholten Zyklen und die möglichen Ursachen wurden ermittelt (z. B. \gls{CTE}-Unterschiede zwischen den Materialien). 

Aufgrund der Bedingungen im Inneren des \gls{STS} ist eine effiziente Temperatur- und Feuchtigkeitsüberwachung und -steuerung erforderlich, um Vereisung oder Wasserkondensation an der Elektronik oder den Siliziumsensoren zu vermeiden. Die wichtigsten Eigenschaften eines geeigneten Sensorkandidaten sind Unempfindlichkeit gegenüber dem Magnetfeld, Toleranz gegenüber ionisierender Strahlung und eine relativ geringe Größe.

Es wurde eine allgemeine Strategie für die Überwachung von Umgebungsparametern im Inneren des \gls{STS} entwickelt, und es wurden potenzielle Sensorkandidaten ausgewählt. Zur Charakterisierung der ausgewählten Sensoren für die relative Luftfeuchtigkeit wurde der entwickelte Kontrollrahmen eingeführt. Ein Probenahmesystem mit einem Keramiksensor und faseroptische Sensoren (\gls{FOS}) wurden als zuverlässige Lösungen für das verteilte Messsystem identifiziert. Zusätzlich werden die industriellen kapazitiven Sensoren als Referenz während der Inbetriebnahme verwendet.

Es wurden zwei verschiedene Ausführungen von \gls{FOS} getestet: ein Hygrometer und 5 Sensoren, die in einem Array gemultiplext sind. Das \gls{FOS}-Hygrometer erwies sich als die zuverlässigere Lösung. Einer der möglichen Gründe für die schlechtere Leistung ist ein relativ geringer Abstand zwischen den nachfolgenden Sensoren (15 cm) und eine dickere Beschichtung. Die Ergebnisse der Untersuchung des Zeitverhaltens zeigten, dass die dünnere Beschichtung von etwa \SI{15}{\micro\metre} einen guten Kompromiss zwischen der Feuchtigkeitsempfindlichkeit und dem Zeitverhalten darstellen sollte. 

Das letzte Kapitel konzentrierte sich auf die Implementierung des auf Containern basierenden Kontrollsystemrahmens für das \gls{mSTS}. Das eingesetzte System erwies sich als zuverlässige Lösung und gewährleistete die Sicherheit des Detektors für fast 1,5 Jahre. Darüber hinaus wurden die Daten zur Leistung der Detektormodule analysiert, und es wurden erhebliche Fortschritte bei der Qualität der Module festgestellt. Die gewonnenen Daten wurden auch zur Schätzung der Gesamtfluenz verwendet, die auf den Veränderungen des Leckstroms beruhte. 

Der entwickelte Rahmen bot eine einzigartige Möglichkeit, verschiedene Versuchsaufbauten zu automatisieren und zu steuern, die wichtige Daten für das \gls{STS} lieferten. Darüber hinaus unterstreicht die Arbeit die Bedeutung eines solchen Systems und skizziert die nächsten Schritte zur Realisierung eines zuverlässigen Detektor-Kontrollsystems für \gls{STS}.
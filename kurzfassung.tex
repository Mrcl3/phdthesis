Das Compressed Baryonic Matter (CBM) ist eines der Kernexperimente in der zukünftigen Facility for Anti-proton and Ion Research (FAIR) in Darmstadt, Deutschland. Das Silicon Tracking System (STS) ist ein zentrales Detektorsystem des CBM, das in einem 1~Tm-Magneten untergebracht ist und bei einer Betriebstemperatur von etwa \SI{-10}{\celsius} arbeitet, um den durch die Strahlung verursachten Volumenstrom in den \SI{300}{\micro\metre} doppelseitigen Silizium-Mikrostreifensensoren niedrig zu halten.

Das STS umfasst acht Tracking-Stationen mit 876 Modulen. Jedes Modul wird kalibriert und getestet, um seine Leistung zu ermitteln. In den nächsten Schritten wird das Modul auf einer Kohlenstoffleiter montiert, und anschließend werden diese Objekte horizontal auf sogenannten C-Frames angeordnet.

Der Schwerpunkt der Arbeit lag auf der Entwicklung eines modularen Steuerungssystems, das für kleine, mittlere und große Versuchsaufbauten eingesetzt werden kann. Dieser Rahmen wurde für Aufbauten verwendet, die eine Fernsteuerung erforderten, wie z. B. die Bestrahlung der Stromversorgungsmodule für das \gls{FEE}, aber auch in laborgestützten Aufbauten, bei denen die Automatisierung und Archivierung erforderlich war (thermische Zyklen der \gls{STS}-Elektronik).

Mit Hilfe der \gls{EPICS}-bezogenen Anwendungen wurde festgestellt, dass das Niederspannungsversorgungsmodul während des \gls{CBM}-Betriebs weiche Fehler von bis zu 9 pro Monat aufweist. Ein solches Verhalten stellt ein Risiko für den Experimentbetrieb dar, da es zu einer Verschlechterung der physikalischen Leistung führen kann, aber auch eine mögliche Gefahr für das \gls{FEE} darstellt. Andererseits würden die \gls{HV}-Kanäle noch häufiger abgeschaltet werden, aber im Falle des \gls{CBM} befinden sie sich weit entfernt vom Experimentierplatz.

Ferner wurde untersucht, wo die Grenzen des \glspl{FEB} in Bezug auf die thermischen Zyklen und die dadurch verursachte mechanische Belastung liegen. Die Ergebnisse dienten als Hinweis auf mögliche Versagensarten des \gls{FEB} am Ende der Lebensdauer des \gls{STS}. Es wurden Versagensarten nach wiederholten Zyklen und mögliche Gründe dafür ermittelt (z. B. \gls{CTE}-Unterschiede zwischen den Materialien). 

Eine weitere Anwendung des entwickelten Rahmens bezog sich auf die Prüfung und Charakterisierung der Feuchtigkeitssensoren. Es wurde eine allgemeine Strategie für die Überwachung der Umgebungsparameter im Inneren des \gls{STS} entwickelt, und es wurden potenzielle Sensorkandidaten ausgewählt. Ein Probenahmesystem mit einem Keramiksensor und \gls{FOS} wurden als zuverlässige Lösungen für das verteilte Sensorsystem identifiziert. Zusätzlich werden die industriellen kapazitiven Sensoren während der Inbetriebnahme als Referenz verwendet.

Das \gls{FOS}-Hygrometer erwies sich als zuverlässigere Lösung im Vergleich zu einem Sensorarray. Einer der möglichen Gründe für die schlechtere Leistung ist ein relativ geringer Abstand zwischen den nachfolgenden Sensoren und eine dickere Beschichtung. Die Ergebnisse der Untersuchung des Zeitverhaltens zeigten, dass die dünnere Beschichtung von etwa 15\,$\mu m$ einen guten Kompromiss zwischen der Feuchtigkeitsempfindlichkeit und dem Zeitverhalten darstellen sollte. 

Kapitel~\ref{chap:msts} konzentrierte sich auf die Hauptimplementierung des auf Containern basierenden Kontrollsystemrahmens für das \gls{mSTS}. Das eingesetzte System erwies sich als zuverlässige Lösung und gewährleistete die Sicherheit des Detektors für fast 1,5 Jahre. Darüber hinaus wurden die Daten zur Leistung der Detektormodule analysiert, und es wurden erhebliche Fortschritte bei der Qualität der Module festgestellt. Die gewonnenen Daten wurden auch zur Schätzung der Gesamtfluenz verwendet, die auf den Veränderungen des Leckstroms beruhte. 
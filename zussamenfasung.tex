Die Facility of Antiproton and Ion Research in Europe (FAIR) [21] ist eine internationale Initiative zur Schaffung einer Forschungseinrichtung für beschleunigerbasierte Forschung. Sie wird einzigartige Forschungsmöglichkeiten in den Bereichen Hadronen- und Kernphysik, Atomphysik, nukleare Astrophysik, Materialforschung, Plasmaphysik und biologische Strahlungsphysik bieten, einschließlich der Entwicklung neuartiger medizinischer Behandlungen und Anwendungen für die Weltraumforschung [22].

Das Compressed Baryonic Matter (CBM) ist eines der Kernexperimente in FAIR. Das Ziel des CBM-Forschungsprogramms ist die Erforschung des QCD-Phasendiagramms im Bereich hoher Baryonendichten mit Hilfe von hochenergetischen Kern-Kern-Kollisionen. Das \gls{STS} ist ein zentrales Detektorsystem des CBM, das in einem 1~Tm untergebracht ist und bei einer Betriebstemperatur von etwa \SI{-10}{\celsius} arbeitet, um den strahlungsinduzierten Volumenstrom in den \SI{300}{\micro\metre} doppelseitigen Silizium-Mikrostreifensensoren niedrig zu halten. Neben dem \gls{STS} verfügt das CBM-Experiment über einen Mikro-Vertex-Detektor (MVD), einen Ring-Imaging-Tscherenkov-Detektor (RICH), Übergangsstrahlungsdetektoren (TRD), einen Flugzeitdetektor (TOF) und einen Projektil-Spektraldetektor (PSD).
\bigbreak
\textbf{Silicon Tracking System}
\bigbreak

Die physikalischen Beobachtungsgrößen bestimmen zusammen mit der vorgesehenen Beschleunigerenergie und Strahlintensität die Anforderungen an das Detektorsystem. Das \gls{STS} ist für die Spurrekonstruktion sowie für die Impulsbestimmung der geladenen Teilchen ausgelegt. Diese Teilchen entstehen bei Kollisionen eines Ionenstrahls mit Energien zwischen \agev{3} und \agev{14} (Protonen \gev{29}) mit einem Target. Bei einer zentralen Au+Au-Kollision entstehen zum Beispiel bis zu 700 Spuren. Das \gls{STS} erstreckt sich über mehr als \SI{1}{\meter} stromabwärts des Targets und wird in einem Volumen von \SI{3}{\cubic\meter} installiert. 

Das \gls{STS} umfasst acht Tracking-Stationen mit 876 Modulen. Jedes Modul wird kalibriert und getestet, um seine Leistung zu ermitteln. In den nächsten Schritten wird das Modul auf einer Kohlenstoffleiter montiert, und anschließend werden diese Objekte vertikal auf sogenannten C-Frames angeordnet. 

Der Schwerpunkt dieser Arbeit lag auf der Entwicklung eines universellen Kontrollsystems, das auf das Silicon Tracking System (\gls{STS}) ausgerichtet ist. Das entwickelte Framework wurde in einer Reihe von Test-Setups verwendet, von kleinen bis hin zu kompletten Implementierungen der Detektor-hardware. Das Ziel dieser Arbeit ist es, das Kontrollsystem zu entwickeln und die mit ihm erzielten Ergebnisse aufzuzeigen. 

\bigbreak
\textbf{Online-Systeme und ihre Aufgaben}
\bigbreak
Für das Hochgeschwindigkeits-CBM-Experiment spielt das triggerlose Datenauslese- und Erfassungssystem eine entscheidende Rolle. Die mit Zeitstempeln versehenen Signale werden ohne Ereigniskorrelation ausgelesen und an eine Hochleistungs-Rechenfarm, den GSI Green IT Cube, übertragen.
In einem ersten Schritt werden die Spuren der geladenen Teilchen aus den Raum- und Zeitinformationen der verschiedenen Detektorsignale rekonstruiert. Anschließend werden die Teilchen identifiziert, wobei sekundäre Zerfallspunkte und Informationen aus RICH oder MUCH, TRD und TOF berücksichtigt werden. Schließlich werden die Teilchen zu Ereignissen gruppiert, die zur Speicherung ausgewählt werden, wenn sie wichtige Observablen enthalten. Parallel dazu werden das Ereignis und seine Ebene anhand von Informationen aus dem PSD charakterisiert.
Ein weiteres wichtiges Online-System wird Experiment Control System (ECS) genannt und besteht aus einer Softwarestruktur, die die Automatisierung, Überwachung und Steuerung der Hardware und der Detektor-Subsysteme ermöglichen soll. Das Detektor-Kontrollsystem (DCS) ist eines der wichtigsten Online-Systeme. 
\bigbreak
\textbf{Detektor-Kontrollsystem}
\bigbreak
Experimente der Hochenergiephysik erfordern komplexe Kontrollsysteme, die für den erfolgreichen Betrieb der Anlage entscheidend sind. Die ordnungsgemäße Implementierung solcher Systeme gewährleistet höhere Sicherheitsmargen und eine bessere Qualität der Datenproduktion. Im Allgemeinen sollte das gesamte System robust, partitioniert, modular, verteilt, mobil und hochverfügbar sein. Ähnliche Themen wurden auch bei der Entwicklung des \gls{STS}-Steuerungssystems berücksichtigt.

Experimental Physics and Industrial Control System (EPICS) und die zugehörigen Toolkits können zur Steuerung großer Experimente oder sogar Strahllinien verwendet werden, aber auch für kleinere Experimente, bei denen nur begrenzte Funktionalitäten benötigt werden (z. B. Datenvisualisierung, Archivierung und Datenbank). Um die Hardware zu testen, die für das endgültige Experiment verwendet werden sollte, wurden viele relativ kleine Forschungs- und Entwicklungsprojekte (einige Hundert Prozessvariablen) gebaut und betrieben. In den beiden folgenden Abschnitten werden die Anwendungen des entwickelten Softwarepakets zur effektiven Steuerung und Datenerfassung in zwei ausgewählten Versuchsaufbauten vorgestellt. Der erste Abschnitt befasst sich mit den Studien zur Bestrahlung der Stromversorgungseinheiten und den Auswirkungen auf die \gls{STS}. Anschließend werden die Ergebnisse der zyklischen thermischen Tests der \gls{STS}-Elektronik vorgestellt und diskutiert. Ziel der thermischen Zyklusstudien war es, die Ggrenzen des Frontend-boards (FEB) zu ermitteln.
\bigbreak
\textbf{Bestrahlung von Stromversorgungsmodulen für Niederspannungselektronik}
\bigbreak
Die Niederspannungsversorgung der Frontend-Elektronik wird in der Experimentierhalle installiert sein. Die Frontend-Elektronik der \gls{STS} wird von etwa 140 Niederspannungsmodulen gespeist. Angesichts der Tatsache, dass einige dieser Module im schlimmsten Fall etwa 40\,mGy/Monat ausgesetzt sein werden, ergibt die Messung, dass etwa 9 SEE pro Monat und Modul auftreten werden. In der Praxis bedeutet dies, dass jedes FEB bis zu 9 Leistungszyklen bei niedrigen Temperaturen pro Monat aushalten muss. Geht man von einem Betrieb von 2 Monaten pro Jahr und einer voraussichtlichen Gesamtbetriebsdauer von 10 Jahren aus, muss die Elektronik mindestens 180 Stromzyklen bei niedrigen Temperaturen von etwa -20 °C standhalten. Andererseits würden die Hochspannung-Kanäle noch häufiger abgeschaltet, aber im Fall der \gls{CBM} befinden sie sich weit entfernt vom Versuchsbereich.
\bigbreak
\textbf{Thermisches Zyklieren von \gls{STS}-Elektronik}
\bigbreak
Insgesamt wurden 12 FEBs untersucht, um die Randbedingungen für den Temperaturbetriebsbereich zu finden. Die durchgeführten thermischen Zyklen führten zur Bestimmung der Grenzen der FEBs, nämlich Ausfälle im Zusammenhang mit den Niederspannungsregler (LDO Reglern). Die Ergebnisse dienten als Hinweis auf mögliche Versagensarten des \gls{FEB} am Ende der \gls{STS} Lebensdauer. Die Versagensarten nach wiederholten Zyklen und die möglichen Gründe wurden ermittelt (z. B. Wärmeausdehnungskoeffizient Unterschiede zwischen den Materialien). 


\bigbreak
\textbf{Lösungen für die Feuchteüberwachung in \gls{STS}}
\bigbreak
Die dritte Testaktivität, die eine wichtige Forschungsarbeit für das \gls{STS} darstellt, bezieht sich auf die Entwicklung und Prüfung verschiedener Sensoren für relative Luftfeuchtigkeit und Temperatur. Aufgrund der rauen Bedingungen im Detektor ist die Auswahl der Feuchtigkeitssensoren eine wichtige Aufgabe, die eine wichtige Aufgabe, die die Betriebssicherheit des Detektors gewährleisten soll. Ein Überblick wurde geschafft über die verschiedenen Lösungen für die Erfassung der Umgebungsparameter und beantwortet die Frage, ob die getestete Technologie den Anforderungen entspricht. Die meisten Anstrengungen wurden unternommen, um faseroptische Sensoren zu charakterisieren und anschließend Sicherheitsanforderungen und Systeme zu entwickeln, die potenziellen Risiken, die z. B. durch eine zu feuchte Umgebung entstehen, entgegenzuwirken.

Die Charakterisierung der Faseroptische Sensoren brachte Informationen über die Vorteile und Grenzen dieser besonderen Technologie mit der Verwendung von Polyimid als empfindlichem Material. Im Prinzip erfüllt das getestete Hygrometer die für die \gls{STS} gestellten Anforderungen. Das verteilte System wird das Probenahmesystem, Faseroptische Sensoren (\gls{FOS}) und kapazitive Sensoren umfassen. Ein Array von Sensoren könnte noch in Betracht gezogen werden, aber der Abstand zwischen den Gittern sollte viel größer als 15 cm sein, um sicherzustellen, dass die Sensoren spannungsfrei verpackt werden können. 

Die Faser-Bragg-Gitter-basierten können \gls{FOS} als Strahlungsfest angesehen werden. Nach [133] können die Sensoren in Strahlungsumgebungen eingesetzt werden, nachdem sie vor der Installation vorbestrahlt werden, um die strahleninduzierte Querempfindlichkeit zu verringern.

Außerdem werden die kapazitiven Industriesensoren neben dem \gls{FOS} eingesetzt. Der Hauptaufgabe besteht darin, sie während der Inbetriebnahme zu verwenden und das \gls{FOS} neu zu kalibrieren, falls die Installation eine zusätzliche Belastung des Gitters verursacht.

Die letzte Technologie, die für das verteilte Messsystem vorgesehen ist, sind die Metalloxid-(Keramik-)Feuchtigkeitssensoren. Dies ist höchstwahrscheinlich die zuverlässigste Lösung, die auch für das Verriegelungssystem verwendet wird. Mehrere Probenahmestellen innerhalb des Detektorgehäuses werden die Spurenfeuchte messen und als Referenz für die beiden anderen Technologien dienen.

\bigbreak
\textbf{\gls{mSTS} als Wegbereiter für das DCS}
\bigbreak

Das \gls{mCBM}-Experiment [159] gilt als FAIR-Experiment der Phase 0 und als Vorläufer des CBM. Die erste \gls{mCBM}-Kampagne fand 2019 nach zwei Jahren Vorbereitungszeit im Detektortestgebiet HTD [160] statt. Der erste \gls{mSTS}-Prototyp wurde zusammen mit mTRD, mTOF, mRICH und mPSD betrieben und bestand aus einer Tracking-Station, die aus vier Detektormodulen (8 FEBs) bestand, die auf zwei Kohlenstoffleitern und anschließend in zwei C-Frames montiert wurden. Die nächste Iteration des \gls{mSTS}-Detektors besteht aus 11 Detektormodulen und wurde zusammengebaut, um ein besseres Verständnis der Komponenten und des Betriebs einer komplexeren Struktur zu erlangen. Die Fertigstellung von 11 Modulen (zusammen mit den Qualitätssicherungsverfahren, \gls{STS} des \gls{STS}-XYTERs und der FEBs), der Auslese- und der Steuerungssoftware, stellt einen wichtigen Meilenstein auf dem Weg zum \gls{STS} dar. Das \gls{mCBM} Experiment, konzentriert sich auf die DCS-Architektur und erschafft einen Einblick in den Betrieb des Detektors. Schließlich werden die mit dem DCS erzielten Ergebnisse erörtert, die Überlegungen zur Verlustleistung, zur Überwachung der Umgebungsbedingungen und zur Bewertung und Berechnung des Leckstroms erortet.

Der erfolgreiche Betrieb des \gls{mSTS}, einschließlich der \gls{DCS}- und \gls{DAQ}-Kette, stellt einen wichtigen Meilenstein auf dem Weg zum Abschluss des \gls{STS}-Projekts dar. Die ersten umfangreichen und erfolgreichen Datenerfassungsaktivitäten mit zwei Tracking-Stationen schlossen die Inbetriebnahme des \gls{mSTS}-Detektors ab. Der Prototyp der \gls{DCS}-Überwachungsschicht von \gls{mSTS} wurde erfolgreich implementiert und bewies, dass das Konzept nicht nur für große Detektoren und Beschleunigeraufbauten, sondern auch für kleinere Experimente äußerst flexibel und nützlich ist. Nach fast zwei Jahren Betrieb hat sich das auf Containern basierende System als zuverlässige, leicht zu wartende Lösung erwiesen. Für das endgültige System werden jedoch noch einige zusätzliche Anwendungen benötigt. Da der \gls{STS}-Detektor wird sehr viel komplexer und anspruchsvoller sein wird, wenn es um die Konfiguration und den Betrieb geht. Bei der endgültigen Einrichtung wird es äußerst wichtig sein, dass sowohl die Hardware als auch die Software verriegelbar sind, um die Sicherheit der Maschine zu gewährleisten.
\bigbreak
\textbf{Fazit}
\bigbreak

Erfolgreiche Implementierungen des EPICS-basierten containerisierten Software-Frameworks wurden in dieser Arbeit vorgestellt. Die Vielseitigkeit und Modularität des Frameworks ermöglichte die Gewinnung von Daten aus kleineren Versuchsanordnungen, z. B. zyklische thermische Belastung der Front-End-Elektronik des Detektors, bis hin zu wesentlich größeren Versuchsanordnungen (z.B. \gls{mSTS}). Die erzielten Ergebnisse lieferten einzigartige Daten, die für die erfolgreiche Integration und den Betrieb des künftigen \gls{STS} verwendet werden sollen.


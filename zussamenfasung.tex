Der Schwerpunkt dieser Arbeit lag auf der Entwicklung eines universellen Kontrollsystems, das auf das Silicon Tracking System (STS) ausgerichtet ist. Das STS ist der wichtigste Tracking-Detektor des Compressed Baryonic Matter Experiments (CBM). Das entwickelte Framework wurde in einer Reihe von Test-Setups verwendet, von kleinen bis hin zu kompletten Implementierungen der Detektor-hardware. Das Ziel dieser Arbeit ist es, die mit dem oben erwähnten Kontrollrahmen erzielten Ergebnisse aufzuzeigen.

Das CBM wird eine der wichtigsten wissenschaftlichen Komponenten der zukünftigen Facility for Antiproton and Ion Research (FAIR) in Darmstadt sein. 


EPICS und die zugehörigen Toolkits können zur Steuerung großer Experimente oder sogar Strahllinien verwendet werden, aber auch für kleinere Experimente, bei denen nur begrenzte Funktionalitäten benötigt werden (z. B. Datenvisualisierung, Archivierung und Datenbank). Um die Hardware zu testen, die für das endgültige Experiment verwendet werden sollte, wurden viele relativ kleine Forschungs- und Entwicklungsprojekte (einige Hundert PVs) gebaut und betrieben. In den beiden folgenden Abschnitten werden die Anwendungen des entwickelten Softwarepakets zur effektiven Steuerung und Datenerfassung in zwei ausgewählten Versuchsaufbauten vorgestellt. Der erste Abschnitt befasst sich mit den Studien zur Bestrahlung der Stromversorgungseinheiten und den Auswirkungen auf die STS. Anschließend werden die Ergebnisse der thermischen Zyklusmessungen für die STS-Elektronik vorgestellt und diskutiert. Ziel der thermischen Zyklusstudien war es, die Betriebsgrenzen des FEB zu ermitteln.